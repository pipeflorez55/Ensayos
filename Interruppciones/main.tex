\documentclass[a4paper]{article}

\usepackage[spanish]{babel}
\usepackage[utf8]{inputenc}
\usepackage{amsmath}
\usepackage{graphicx}
\usepackage[colorinlistoftodos]{todonotes}
 
 
\title{INTERRUPCIONES}
\author{Andrés Felipe Flórez Gil \\ CC. 1017269766}

\date{\today}

\begin{document}
\maketitle

En programación una interrupción es un mecanismo a través del cual se envía una señal al microprocesador para indicarle que debe detener la ejecución del programa actual para ejecutar otro proceso prioritario. Puede suceder como un evento asíncrono y una vez que se ejecuta se devuelve al programa anterior.[1]

Las interrupciones se pueden clasificar según cómo se genere la señal para ejecutar la interrupción y existen diferentes tipos. De manera general se puede clasificar con síncronas y asíncronas, siendo las primeras provocadas por la ejecución de un programa dentro de la CPU, son internas al programa; y las segundas son provocadas por eventos externos al programa notificando al sistema operativo a cerca de algún cambio en el ambiente de operatividad y ejecutando una interacción para tomar una decisión respecto al cambio que no se pudo tomar autónomamente por el sistema operativo. De manera específica  se clasifica las interrupciones como:
-Interrupciones de Hardware:
-Interrupciones de Software:



\\

Las interrupiones ingresan como una señal proveniente del periférico la cual le dice a nuestro micrcontrolador que debe parar su ejecucion en curso al terminar la ultima instruccion en ejecuccion despues de esto guarda el estado actual (PC y REGISTROS)en la pila y se obtiene el pc de la subrutina correspondiente a esta interrupcion en caso de llegar varias interupciones a la vez se maneja un sistema de prioridad el cual va decidir cual subrutina seguira en curso, despues de obtener el pc de la subrutina correspondiente se quita el bit correspondiente a la entrada a modo de interrupcion para no quedar en ciclo infinito  y procede a realizar la maquina de estado con las insstrucciones de la subrutina en marcha al acabar se ejecuta la instruccion RTI la cual vuleve a cargar el punto en el cual estaba el progama (PC y REGISTROS) desde el stack, de esta manera el microcontrolador maneja las interrupciones entrantes.
\\
Las interrupciones implementadas desde de software  son usada en los programas para hacer uso de perifericos o funciones del sistema operativo estos programas se pueden auto interrumpir para dar paso a la lectura de un disco o la ejecucion de otro proceso en el sistema operativo estas interrupciones se pueden hacer tambien llamando interrupiones directamente a la bio, estas interrupciones en software son mucho mas veloces cuando son implementadas directamente en la bios mediante ensamblador pero  necesitando conocimiento mas profundos en programacion de bajo nivel con ensablador en cambio las interrupciones que son implementadas sobre el sistema operativo son masss lentas pero son mas comodas y faciles de usar, estas interrupciones echas sobre el sistema operativo son echas con lenguajes de programacion de alto nivel como puede ser c pero tambien van a cambiar dependiendo del compilador y el hadware ya que si cambia el hardware el protocolo de instrucciones puede cambiar tambien el compilador el cual la ayuda entre lo que escribimo y como lo lleva a bajo nivel todo esto pero para solucionar estos problemas en gran parte se hace uso de convenios entre los compiladores para hacer una base en lenguaje de alto nivel para tratar estas interrupciones en este lenguaje esto mismo sucede con los diferentes lenguajes de progrmacion los cuales algunos son mas eficiente o faciles para trabajar a la hora de interrupciones que otros


\begin{thebibliography}{X}
\bibitem{Alc}http://www.ciens.ucv.ve:8080/genasig/sites/organizacion-del-comp-II/archivos/Interrupciones.pdf


\bibitem{Alc}http://index-of.co.uk/Winasm-studio-tutorial/manual2-8086.pdf

\bibitem{Alc}https://es.wikipedia.org/wiki/Interrupci%C3%B3n

\bibitem{Alc}https://aprendiendoarduino.wordpress.com/2016/11/13/interrupciones/
\bibitem{Alc}https://sites.google.com/site/lgiao2018/unidad-1/1-4-el-concepto-de-interrupciones
\bibitem{Alc}https://books.google.com.co/books?id=I3w5DwAAQBAJ&pg=PA11&dq=microprocesadores&hl=es&sa=X&ved=2ahUKEwjCu7PKiqvqAhWIdN8KHSqlA5AQ6AEwBHoECAUQAg#v=onepage&q=microprocesadores&f=false
\bibitem{Alc}https://www.slideshare.net/matrix1979/interrupciones-53999787
\bibitem{Alc}http://www.fdi.ucm.es/profesor/jjruz/WEB2/Temas/Curso05_06/EC9.pdf
\bibitem{Alc}http://mimosa.pntic.mec.es/~flarrosa/raton.pdf
\bibitem{Alc}http://www.angelfire.com/al4/pc/int.htm
\end{thebibliography}
\end{document}
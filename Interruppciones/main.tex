\documentclass[a4paper]{article}

\usepackage[spanish]{babel}
\usepackage[utf8]{inputenc}
\usepackage{amsmath}
\usepackage{graphicx}
\usepackage[colorinlistoftodos]{todonotes}
 
 
\title{INTERRUPCIONES}
\author{Andrés Felipe Flórez Gil \\ CC. 1017269766}

\date{\today}

\begin{document}
\maketitle

En programación una interrupción es un mecanismo a través del cual se envía una señal al microprocesador para indicarle que debe detener la ejecución del programa actual para ejecutar otro proceso prioritario. Puede suceder como un evento asíncrono y una vez que se ejecuta se devuelve al programa anterior.[1]




\begin{thebibliography}{X}
\bibitem{Alc}http://www.ciens.ucv.ve:8080/genasig/sites/organizacion-del-comp-II/archivos/Interrupciones.pdf


\bibitem{Alc}http://index-of.co.uk/Winasm-studio-tutorial/manual2-8086.pdf

\bibitem{Alc}https://es.wikipedia.org/wiki/Interrupci%C3%B3n

\bibitem{Alc}https://aprendiendoarduino.wordpress.com/2016/11/13/interrupciones/
\bibitem{Alc}https://sites.google.com/site/lgiao2018/unidad-1/1-4-el-concepto-de-interrupciones
\bibitem{Alc}https://books.google.com.co/books?id=I3w5DwAAQBAJ&pg=PA11&dq=microprocesadores&hl=es&sa=X&ved=2ahUKEwjCu7PKiqvqAhWIdN8KHSqlA5AQ6AEwBHoECAUQAg#v=onepage&q=microprocesadores&f=false
\bibitem{Alc}https://www.slideshare.net/matrix1979/interrupciones-53999787.
\end{thebibliography}
\end{document}
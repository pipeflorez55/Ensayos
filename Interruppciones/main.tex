\documentclass[a4paper]{article}

\usepackage[spanish]{babel}
\usepackage[utf8]{inputenc}
\usepackage{amsmath}
\usepackage{graphicx}
\usepackage[colorinlistoftodos]{todonotes}
 
 
\title{Interrupciones}
\author{Andrés Felipe Flórez Gil \\ CC. 1017269766}

\date{\today}

\begin{document}
\maketitle

En programación una interrupción es un mecanismo a través del cual se
envía una señal al microprocesador para indicarle que debe detener la
ejecución del programa actual para ejecutar otro proceso prioritario.
Puede suceder como un evento asíncrono y una vez que se ejecuta se
devuelve al programa anterior.[1]\\
Hablando históricamente de las interrupciones se puede destacar el
avance
en ejecución y manejo que han tenido con el paso del tiempo. Existe una
técnica antigua de interrupción llamada 'Polling', cuyo mecanismo se
basa
en un sondeo continuo, donde el microcontrolador es el encargado de
verificar los estados de los eventos, lo que trae como consecuencia que
no puedan ser monitoreados los datos constantemente. Esta técnica no
resulta muy útil por ejemplo, en una aplicación multitareas donde lo que
se busca es versatilidad para que no se vean interrumpidos todos los
programas en ejecución. Debido a esto fue necesario encontrar nuevas
maneras de ejecutar estas operaciones y por ello la importancia de la
implementación de otros mecanismos de interrupciones como el Daisy-chain
que se refiere a conexión en cadena de diferentes dispositivos en orden
decreciente de prioridad. La señal emitida sólo pasará a los
dispositivos
siguientes en el orden establecido si el anterior no solicita el
servicio
del procesador para llevar a cabo la nueva operación. Es evidente como
el
avance en materia de procesamiento de interrupciones para hacer frente a
nuevas necesidades marca una importante evolución y mejora de los
programas, como tenemos la capacidad de cada vez hacerlos más eficaces y
mejorar su experiencia. [3][5] \\

Las interrupciones se pueden clasificar según cómo se genera la señal
para ejecutar la interrupción y existen diferentes tipos. De manera
general se puede clasificar con síncronas y asíncronas, siendo las
primeras provocadas por la ejecución de un programa dentro de la CPU,
son

internas al programa; y las segundas son provocadas por eventos externos
al programa notificando al sistema operativo a cerca de algún cambio en
el ambiente de operatividad y ejecutando una interacción para tomar una
decisión respecto al cambio que no se pudo tomar autónomamente por el
sistema operativo. De manera específica  se clasifica las interrupciones
como: \\

-Interrupciones internas de Hardware:
Las interrupciones internas son generadas por ciertos eventos que
ocurren durante la ejecución. Este tipo de interrupciones son manejadas
en un cien por ciento  por el hardware y, no es posible modificarlas a
través del software. \\

-Interrupciones externas de Hardware:
Son enviadas por dispositivos periféricos o por coprocesadores. No es
posible desactivar este tipo de interrupciones y, no son mandadas
directamente a la CPU sino a circuito integrado especializado únicamente para tratar estas interrupciones.\\

-Interrupciones de Software:
Este tipo de interrupciones pueden ser activadas directamente por el ensamblador o generadas por un mismo programa en ejecución.\\

Las interrupciones ingresan como una señal proveniente del periférico la
cual le dice a nuestro microcontrolador que debe parar su ejecución en
curso al terminar la última instrucción en ejecución después de esto
guarda el estado actual (PC y REGISTROS)en la pila y se obtiene el pc de
la subrutina correspondiente a esta interrupción en caso de llegar varias
interrupciones a la vez se maneja un sistema de prioridad el cual va
decidir cual subrutina seguirá en curso, después de obtener el pc de la
subrutina correspondiente se quita el bit correspondiente a la entrada a
modo de interrupción para no quedar en ciclo infinito  y procede a
realizar la máquina de estado con las instrucciones de la subrutina en
marcha al acabar se ejecuta la instrucción RTI la cual vuelve a cargar
el punto en el cual estaba el programa (PC y REGISTROS) desde el stack,
de esta manera el microcontrolador maneja las interrupciones entrantes.
\\

Las interrupciones implementadas desde de software  son usadas en los programas para hacer uso de periféricos o funciones del sistema operativo estos programas se pueden auto interrumpir para dar paso a la
lectura de un disco o la ejecución de otro proceso en el sistema
operativo. Estas interrupciones se pueden hacer también llamando interrupciones directamente a la bio, estas interrupciones en software son mucho más veloces cuando son implementadas directamente en la bios mediante ensamblador pero  necesitando conocimiento más profundos en programación de bajo nivel con ensamblador, por el contrario las
interrupciones que son implementadas sobre el sistema operativo son más lentas pero son más cómodas y fáciles de usar, estas interrupciones hechas sobre el sistema operativo son hechas con lenguajes de programación de alto nivel como puede ser C pero también van a cambiar dependiendo del compilador y el hardware, ya que, si este último cambia, el protocolo de instrucciones también puede cambiar el compilador el cual ayuda entre lo que se escribe y cómo lo lleva a bajo nivel. Para
darle solución  a estos procesos  en gran parte se hace uso de convenios entre los compiladores para hacer una base en lenguaje de alto nivel para tratar estas interrupciones. Esto mismo sucede con los diferentes lenguajes de programación, en los cuales se observa que algunos son más eficientes que otros o fáciles para trabajar con interrupciones.\\

En conclusión las interrupciones y el procesamiento de las mismas han ayudado y siguen ayudando ampliamente a la informática a través de diferentes mecanismos, atendiendo distintos tipos de necesidades que se presentan en determinada situación. Una de sus más destacadas utilidades es que ayudan a disminuir el tiempo de espera en el cual se puede leer una señal externa al programa comparado al sistema anterior. Con esto trae como beneficio, con gran cantidad de periféricos, mejorar la capacidad de responder de manera veloz los requerimientos de servicios de los programas. 

\title{Bibliografia}

\bibitem{Alc} [1] http://www.ciens.ucv.ve:8080/genasig/sites/organizacio n-del-comp-II/archivos/Interrupciones.pdf
\bibitem{Alc} [2]http://index-of.co.uk/Winasm-studio-tutorial/manual2-8086.pdf
\bibitem{Alc} [3]https://es.wikipedia.org/wiki/Interrupcion

\bibitem{Alc} [4]https://aprendiendoarduino.wordpress.com/2016/11/13/interrupciones/
\bibitem{Alc} [5] http://www3.fi.mdp.edu.ar/electrica/opt_archivos/arduino/Manejo_de_Interrupciones.pdf

\bibitem{Alc} [6] https://books.google.com.co/books?id=I3w5DwAAQBAJ&pg=P
A11&dq=microprocesadores&hl=es&sa=X&ved=2ahUKEwjCu7PKiqvqAhWIdN8KHSqlA5A
Q6AEwBHoECAUQAg#v=onepage&q=microprocesadores&f=false

\bibitem{Alc} [7] http://www.fdi.ucm.es/profesor/jjruz/WEB2/Temas/Curso05_06/EC9.pdf

\bibitem{Alc} [8] http://mimosa.pntic.mec.es/~flarrosa/raton.pdf

\bibitem{Alc} [9]http://www.angelfire.com/al4/pc/int.htm
\end{Bibliografia}
\end{document}
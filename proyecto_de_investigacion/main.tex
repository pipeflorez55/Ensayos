\documentclass[a4paper]{article}

\usepackage[spanish]{babel}
\usepackage[utf8]{inputenc}
\usepackage{amsmath}
\usepackage{graphicx}
\usepackage[colorinlistoftodos]{todonotes}
 
 
\title{Receta para la informática moderna}
\author{Andrés Felipe Flórez Gil \\ CC. 1017269766}

\date{\today}

\begin{document}
\maketitle

Las primeras nociones de la informática que conocemos hoy en día tuvieron inicio en el siglo XX motivadas por discusiones que se empezaron a plantear en el mundo matemático, de las cuales surgieron soluciones indispensables e importantes para lo que fue un posterior desarrollo de la sociedad con base en la ciencia y la computación.\\


La idea de computación se empezó  a formar en el marco del final del concepto de una matemática infalible. Todo  comenzó con con  Kurt Gödel y Alan Turing. Gödel probó formalmente que ningún sistema podría ser a la vez consistente, recursivo y completo, queriendo decir que los axiomas de la matemáticas podrían llevar a paradojas sin solución si se tratan de resolver en el mismo marco creado por ellos. Después de esto Turing para demostrar parte de lo planteado por gödel creó la idea de una máquina la cual seria programable y por medio de esto pudiera resolver algún algoritmo requerido. Con esta máquina y el teorema de "la parada" que era el mismo problema de incompletitud de gödel pero llevado a su máquina, planteó que no todo tiene solución, dado que algunos problemas crean ambigüedades que generan bucles infinitos al querer resolverlos en su mismo marco. Todas estas demostraciones llevaron a planteamientos e ideas que dieron base a la computación moderna. De esta manera, las acciones de Turing crearon una especie de puente de comunicación entre los planteamientos de la matemática de la época y lo que empezaron a ser primeras consideraciones de una nueva era de avances con base en "la computación". Esto marcó un agigantado paso en el desarrollo de avances científicos y tecnológicos para la sociedad, y en la creación de las herramientas útiles para llevarlos a cabo.\\ 


Los aportes de Alan Turing revolucionaron la manera de ver la vida en la época, como lo menciona B. Jack: “Alan Turing cambió el mundo”, lo cual queda evidenciado con los inventos y avances que se hicieron posteriormente en la historia a raíz de los aportes dados por este hombre, las pruebas de este hecho son evidentes, como son la creación de una máquina programable capaz de resolver algoritmos, la idea de cómputo con la cual podemos llevar un problema a un algoritmo y después programarlo en una máquina, y la clasificación de problemas en solubles o no algorítmicamente. Por otro lado, predijo un aspecto del futuro de la computación referente al término que se usa para decir que una computadora se "colgó" lo que quiere decir que llegó al teorema de la parada, la máquina llega a un problema que no puede computar y queda en ciclo infinito.\\


Alan Turing causó un impacto positivo en la educación, sobretodo en la educación superior con sus aportes científicos, ya que brindó las bases para la creación de herramientas que fueron importantes para permitir la transformación de la educación en el tiempo, y de esta manera mejorar la calidad de pedagogías y aprendizaje, ejemplo de estas herramientas esenciales son un computador, o un teléfono móvil. Este impacto ayuda a que hoy en día muchos estudiantes por todo el mundo se preparen de la mejor manera y pongan al servicio de la sociedad sus habilidades en esta área del conocimiento para la solución de problemas complejos de actualidad. Como dijo el genio de la computación Alan Turing: "Un hombre provisto de papel, lápiz y goma, y con sujeción a una disciplina estricta, es en efecto una máquina de Turing universal.", en efecto si un hombre con las herramientas mencionadas es capaz de igualar la eficiencia de la máquina de Turing, con objetos más avanzados en tecnología definitivamente no tendrá límites. Los aportes de Turing también quedan evidenciados en el ámbito político debido a que fue capaz de desbloquear el código de la máquina enigma utilizado por los submarinos alemanes en el Atlántico. Su trabajo está considerado clave para el final de la II Guerra Mundial.\\

En conclusión, con justas razones Alan Turing es considerado el padre de la computación y un referente muy importante para la historia actual, importante también para el desarrollo de la tecnología en el tiempo. También se puede concluir que hay que aprovechar las herramientas que tenemos en la actualidad para generar grandes cambios al servicio de la sociedad, pero que también para un gran avance es necesario cuestionar las bases y teorías actuales, básicamente todo lo que esté a nuestro al rededor. Los aportes de Alan Turing así como el de muchos otros científicos informáticos fueron los ingredientes ideales para configurar la vida que vivimos hoy, en la cual muchos aspectos importantes como redes sociales, pagos bancarios, globalización del comercio, entre otros, dependieron en su creación de la innovación tecnológica que se ha venido dando con el transcurrir de los años; el legado de su conocimiento ha transformado la informática moderna y ahora es el turno de las generaciones jóvenes actuales de hacer historia y dejar una huella. 

\begin{thebibliography}{X}
\bibitem{Alc}Diario El Mundo. Recuperado de Diario El Mundo: 
https://www.elmundo.es/tecnologia/2013/12/24/52b94ecd268e3e89648b456f.html
\bibitem{Alc}Madrimasd: 
http://www.madrimasd.org/blogs/matematicas/2012/10/15/135043
\bibitem{Alc}Diario La Vanguardia. Recuperado de Diario La Vanguardia: 
https://www.lavanguardia.com/historiayvida/historia-contemporanea/20180611/47312986353/que-aporto-a-la-ciencia-alan-turing.html
\bibitem{Alc}OpenMind BBVA. Recuperado de OpenMind BBVA: 
https://www.bbvaopenmind.com/tecnologia/inteligencia-artificial/alan-turing-y-el-sueno-de-la-inteligencia-artificial/
https://www.bbvaopenmind.com/ciencia/matematicas/asi-termino-el-sueno-de-las-matematicas-infalibles/
\end{thebibliography}
\end{document}
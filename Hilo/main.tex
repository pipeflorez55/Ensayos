\documentclass[a4paper]{article}

\usepackage[spanish]{babel}
\usepackage[utf8]{inputenc}
\usepackage{amsmath}
\usepackage{graphicx}
\usepackage[colorinlistoftodos]{todonotes}
 
 
\title{Hilos}
\author{Andrés Felipe Flórez Gil \\ CC. 1017269766}

\date{\today}

\begin{document}
\maketitle
En programación un hilo es similar a un proceso en que los dos cumplen tareas  en paralelo con otras secuencia. En el 2000, sistemas operativos como linux no diferenciaban hilos y procesos. Hoy en dia un hilo es una tarea muy pequeña y el proceso puede llegar a estar formado por uno o más hilos, del cuales se dice que está activo si uno de sus hilos sigue activo. Esta diferenciación entre hilos y procesos lleva a confusiones muy amenudo ya que en un principio su diferenciación era muy mínima. Actualmente con la implementación de hilos que hay, se puede multiplicar la eficiencia de los procesadores llevando así al conocimiento de hilos por parte de personas amantes de la computación debido a que es un tema importante a la hora de escoger un procesador. Aunque los hilos sean más nombrados en el ámbito de la computación por procesadores, en los microprocesadores también efectuaron un cambio importante ya que la misma teoría es aplicada a ellos de tal manera que la eficiencia también aumenta.\\

Para explicar cómo funciona un microprocesador con multihilo podemos usar como ejemplo una fila en un supermercado en la cual el cajero es muy veloz y llega un cliente al cual se le olvido un pedido. En el tiempo de antes el cajero(microprocesador) tenía que esperar a que le trajeran el pedido que le faltaba al cliente para seguir con la siguiente tarea, pero con el multihilo sucede que cliente que se le olvidó el pedido es puesto a un lado en espera y cómo hay 2 filas vez de una, se pasa un cliente de la otra fila hasta que traigan el pedido del cliente en espera. De esta manera se hace un uso más eficiente del procesamiento ya que de nada sirve una cinta transportadora super rapida si no se le pasan productos para procesar.\\

Todo lo relacionado a los hilos puede sonar una maravilla para la programación,  pero sus orígenes no son  muy recientes. Su historia viene al mismo ritmo que la de la programación y computación, dado que los primeros indicios de hilos se puede retomar hasta Djistrak en 1965. Aunque en esa época no se tenía la noción de hilos se manejaba el término de proceso el cual inició con la computación que se podría ver como un solo hilo y el uso de multihilos se puede retomar a prototipos de IBM aunque no se tiene información si algun compilador lo llegó a aplicar, luego de esto siguió unix con un proceso secuencial pero con direccionamiento virtual. La teoría recolectada de hilos sirvió bastante en la comunicación llevando así a su mejoramiento hasta nuestra época.\\
\\\\

Los hilos se pueden diferenciar en hilos a nivel de usuario y hilos a nivel de kernel, llevando varias manera de iteración entre estos dos tipos de hilos de las cuales son modelos Mx1,1x1,MxN, dando a entender la cantidad de hilos de usuarios llevados a hilos de kernel, lo cual es posible gracias a que un hilo de usuario no requiere soporte del SO. Estos hilos de usuarios ya están implementados en algunos lenguajes de programación como Java o Delphi, pero en otros como C# son necesarias las librerías correspondientes a sistemas operativos para poder hacer uso de hilos.\\


En conclusión, los hilos es uno de los mejores mecanismos y de los más eficientes para manejar procesamientos hoy en día. Los hilos continúan siendo estudiados a la par que la computación, llegando así a estar presente a nivel de hardware en todos los procesadores actuales, y a nivel de software teniendo librerías para un uso correcto, aportando más eficiencia y vía libre para programas más complejos y de mayor necesidad de procesamiento.\\


\title{Bibliografia}

\bibitem{Alc} [1] http://www.serpentine.com/blog/threads-faq/the-history-of-threads/
\bibitem{Alc} [2]https://www.fing.edu.uy/tecnoinf/mvd/cursos/so/material/teo/so05-hilos.pdf
\bibitem{Alc} [3]http://bibing.us.es/proyectos/abreproy/11320/fichero/Capitulos%252F13.pdf

\bibitem{Alc} [4]https://es.wikipedia.org/wiki/Hilo_(inform%C3%A1tica)#Hilos_a_nivel_de_n%C3%BAcleo_(KLT)

\end{Bibliografia}
\end{document}
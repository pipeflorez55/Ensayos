\documentclass[a4paper]{article}

\usepackage[spanish]{babel}
\usepackage[utf8]{inputenc}
\usepackage{amsmath}
\usepackage{graphicx}
\usepackage[colorinlistoftodos]{todonotes}
 
 
\title{Hilos}
\author{Andrés Felipe Flórez Gil \\ CC. 1017269766}

\date{\today}

\begin{document}
\maketitle
En programacion un hilo es similiar a un proceso en que los dos cumples tareas  en paralelo con otras secuencia.En el 2000 sistemas operativos como linux no diferenciabas hilos y procesos hoy en dia un hilo es una tarea muy pequeña y el proceso puede llegar a estar formado por uno o mas hilos y el cual se dice que esta activo si uno de sus hilos sigue activo. Esta difrenciacion entre hilos y procesos lleva a confuciones muy amenudo ya que en un principio su diferenciacion era muy minima, hoy en dia con la implementacion de hilos que hay se puede multiplicar las eficiencia de los procesadores llevando asi al conocimiento de hilos por parte de personas amantes de la computacion al ser un tema importante a la hora de escoger un procesador. Aunque los hilos sean mas nombrados en el ambito de la computacion por procesadores en los microprocesadores tambien efectuaron un cambio importante ya que la misma teoria es aplicada a ellos de tal manera que la eficiencia tambien fue aumentada.\\

En si como funciona microprocesador con multihilo podemos usar como ejemplo una fila en un supermecado en el cual el cajero es muy veloz y llega un cliente el cual se le olvido un pedido, en el tiempo de antes el cajero(microprocesador) tenia que esperar a que le trajeran el pedido que le faltaba al cliente para seguir con la siguiente tarea, con ell multihilo lo que pasa es que el cliente que se le olvido el pedio es puesto a un lado en espera y como en ves de haber 2 filas en vez de una se pasa un cliente de la otra fila hasta que traigan el pedido del cliente en espera, de esta manera se hace un uso mas eficiente del procesamiento ya que de nada sirve una cinta transportadora super rapida si no se le pasan productos de procese.\\

Todo lo consecuente a los hilos puede sonar una maravilla para la programacion  pero como emepezo todo no es muy reciente ya que su historia viene con la misma historia de la programacion y computacion, dado que los primeros indicios de hilos se puede retomar hasta Djistrak en 1965 aun que en esa epoca no se tenia la nocion de hilos se manejaba el termino de proceso el cual inicio con la computacion que se podria ver como un solo hilo y el uso de multi hilos se puede retomar a prototipos de IBM aunque no se tiene informacion si algun compilador lo llego a aplicar, de "aca siguio unix con un proceso secuencial pero con direcionamiento virtual". La teoria recolectada de hilos sirvio bastante en la comunicacion y llevando asi a su mejoramiento hasta hoy en dia.\\
\\\\

Los hilos se pueden diferenciar en hilos a nivel de ususario y hilos a nivel de kerne, llevando varias manera de interacion entre estos dos tipos de hilos de las cuales son modelos Mx1,1x1,MxN, dando a entender la cantidad de hilos de usuarios llevados a hilos de kernel esto se puede dar ya que un hilo de usuario no requiere soporte del SO, estos hilos de usuarios ya estan implementados en algunos lenguajes de programacion como Java o Delphi pero en otros commo C# son necesarias librerias correspondientes a sistemas operativos para poder hacer uso de hilos.\\


Podemos llegar a la conclucion en que los hilos es de las maneras mas eficiente de manerjar proocesamientos hoy en dia y la cual viene siendo estudia a la par que la computacion llegando asi a estar presente a nivel de hadware en todos los rocesadores actuales y a nivel de sofware teniendo librerias para sus usos correcto dando asi mas eficiencia via libre para programas mas complejos y de mayo necesidad de procesamiento.\\

     





\title{Bibliografia}

\bibitem{Alc} [1] http://www.serpentine.com/blog/threads-faq/the-history-of-threads/
\bibitem{Alc} [2]https://www.fing.edu.uy/tecnoinf/mvd/cursos/so/material/teo/so05-hilos.pdf
\bibitem{Alc} [3]http://bibing.us.es/proyectos/abreproy/11320/fichero/Capitulos%252F13.pdf

\bibitem{Alc} [4]https://es.wikipedia.org/wiki/Hilo_(inform%C3%A1tica)#Hilos_a_nivel_de_n%C3%BAcleo_(KLT)

\end{Bibliografia}
\end{document}